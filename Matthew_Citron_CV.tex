\documentclass[9pt]{res} % default is 10 pt
\usepackage{helvetica} % uses helvetica postscript font (download helvetica.sty)
%\usepackage{newcent}   % uses new century schoolbook postscript font 
%\setlength{\textheight}{9.5in} % increase text height to fit resume on 1 page
\newsectionwidth{0pt}  % So the text is not indented under section headings
\usepackage{enumitem} %Package to formag the list
\setitemize{itemsep=0.05cm, leftmargin=*}
\usepackage[margin=0.7in]{geometry}


\begin{document} 
 
\name{Matthew Citron} % the \\[12pt] adds a blank line after name
\vspace{0.2cm}
\address{ \\44 Rue de la Combette\\01630 St Genis-Pouilly\\ France\\	matthew.citron09@imperial.ac.uk}
%\address{{\bf PERMANENT ADDRESS} \\ 53 Peel Street Glasgow
 %       North Lanarkshire G11 5LX \\ 07728209851}
 
                                             
\begin{resume}
                                               
 
\section{OBJECTIVE}
\vspace{0.2cm}
    Particle physics inspired my desire to study physics. My interest has been enriched through the experience of working for a PhD in High Energy Physics. My objective is therefore to continue contributing to research in this fundamental field after I finish my PhD.
   
\section{EDUCATION} 

\vspace{0.2cm}

\noindent 2013-2017 IMPERIAL COLLEGE LONDON \\
PhD in High Energy Physics, ongoing.

\noindent 2012-2013 IMPERIAL COLLEGE LONDON \\
MSc, Quantum Fields and Fundamental Forces, Distinction.

\noindent 2009-2012 IMPERIAL COLLEGE LONDON \\
BSc 3YFT, Physics with Theoretical Physics, First Class Honours.

\noindent 2003-2009 HYNDLAND SECONDARY SCHOOL \\
Mathematics A, Physics A, Chemistry A (Advanced High). \\
Mathematics A, Physics A, Chemistry A, English A, History A (Higher).\\
Level of qualification: SQA Higher + Advanced Higher, equivalent EQF levels 4+5. \\
 
\section{EXPERIENCE} 
\vspace{0.2cm}
PHD IN HIGH ENERGY PHYSICS, 01/10/2013 - present \\
\vspace{0.1cm}
High Energy Physics Group, Imperial College London 
   \begin{itemize} \itemsep -2pt  % reduce space between items
   \item Worked to develop the jet algorithm for the Stage 2 upgrade of the CMS level-one trigger, particularly focusing on novel methods of pile-up subtraction.
   \item Part of core team on an all hadronic search for new physics that quickly interpreted the data from the initial 13TeV run of the LHC (06/15-11/15) to deliver one of the first public SUSY results for the jamboree event on 14/12/15. 
   \item Work includes developing the statistical analysis of the search as well as looking at new variables and strategies to optimise sensitivity to a wide range of models.
   \item Worked on developement of python framework for analysis. 
   \item Additionally work with a phenomenology collaboration (MasterCode) to determine the impact of direct searches at CMS and ATLAS on the allowed parameter space of GUT scale models.
   \item Undertook trigger shifts for CMS.
   \item Moved to France to be based at CERN. 
 \end{itemize}

UNDERGRADUATE RESEARCH OPPORTUNITIES PROGRAMME, 30/06/2012 - 30/09/2012 \\
\vspace{0.1cm}
High Energy Physics Group, Imperial College London
   \begin{itemize} \itemsep -2pt  % reduce space between items
   \item Recast searches for supersymmetry at the LHC for use in a scan of GUT scale models to determine the impact of these searches on the SUSY parameter space.
   \item Gained experience with event generation (PYTHIA) as well as fast detector simulation (DELPHES).
   \item Additionally, studied the decay times of long lived staus as part of a proposal for a new search.
   \item Both parts of the project form sections of publications in which I am a co-author.
 \end{itemize}

DAAD SCHOLARSHIP, 26/06/2011 - 19/09/2011\\
\vspace{0.1cm}
Max-Planck-Institut fur Kernphysik (MPIK), Heidelberg
   \begin{itemize}\itemsep -2pt  % reduce space between items
   \item Assist the EBIT (electron beam ion trap) group at MPIK with research into highly charged ions.
   \item Experience in experimental techniques and data analysis from work on using an EBIT as a Penning trap.
   \end{itemize}

WORK PLACEMENT, 02/08/2010 - 02/09/2010\\
\vspace{0.1cm}
Scottish Universities Environmental Research Centre (SUERC)
   \begin{itemize} \itemsep -2pt  % reduce space between items
   \item While still at secondary school, gained experience in experimental techniques and data analysis investigating the luminescent properties of different materials with the aim of using luminescence as an environmental dosimeter.
 \end{itemize}

\section{CONFERENCES AND PUBLIC SPEAKING}
 \begin{itemize}
   \item I have presented work carried out by myself and colleagues on many occasions to the supersymmetry (SUSY) group at CERN.
   \item Presented results of all-hadronic SUSY search at meeting of all UK institutes working on CMS (CMS UK) on two occasions.
 \end{itemize}

\section{COMPUTING}
 \begin{itemize}
   \item I am experienced with programming in C++ and Python.
   \item I am experienced in using Bash shell script and UNIX based operating systems.
   \item Proficient in MatLab and LabView.
   \item Proficient in Microsoft Office and LaTeX.
 \end{itemize}

\section{ORGANISATION AND TEAMWORK}
\begin{itemize}
   \item Work in an international team at CERN to search for SUSY.
   \item Part of the MasterCode collaboration of theorists and experimentalists.
   \item Member of the CMS collaboration.
\end{itemize}

\section{ACTIVITIES} 
%\vspace{0.1cm}
\begin{itemize}
\item I am an enthusiastic rock climber, runner and skier.
\end{itemize}

\section{AWARDS AND SCHOLARSHIPS}
\begin{itemize}
\item STFC funding for PhD (2014-2017).
\item UROP funding to carry out research in Imperial College (2012 and 2013).
\item DAAD RISE Scholarship to carry out research in Germany for 10 weeks (2011).
\item School prizes for physics, maths and chemistry (2009).
\end{itemize}

\section{PUBLICATIONS}
CMS Collaboration. \textit{Search for new physics in final states with jets and missing transverse momentum in 13 TeV pp collisions with the alphaT variable}. CDS 2016.\\
Kreis et al. \textit{Run 2 upgrades to the CMS Level-1 calorimeter trigger}. Journal of Instrumentation, Volume 11, January 2016.\\
Buchmueller et al. \textit{Collider interplay for supersymmetry, higgs and dark matter}. EPJ C October 2015.\\
Buchmueller et al. \textit{Supersymmetric dark matter after LHC run 1.} EPJ C October 2015.\\
de Vries et al. \textit{The pMSSM10 after LHC run 1.} EPJ C September 2015.\\
Buchmueller et al. \textit{The NUHM2 after LHC run 1.} EPJ C September 2014.\\
Citron et al. \textit{End of the CMSSM coannihilation strip is nigh.} Phys. Rev. D Febuary 2013.\\
Buchmueller et al. \textit{The CMSSM and NUHM1 in light of 7 TeV LHC, $Bs \rightarrow \mu\mu$ and XENON100 data}. EPJ C August 2012.



%\section{CONFERENCE TALKS}
%%Add publications?


\section{REFERENCES}
\vspace{0.2cm}
%first column
\begin{minipage}[t]{0.5\textwidth}
Dr Oliver Buchmueller\\
Lecturer in Physics\\
Imperial College London\\
Email: Oliver.Buchmueller@Cern.ch
\end{minipage}
%second column
\begin{minipage}[t]{0.5\textwidth}
Prof. John Ellis\\
Clerk Maxwell Professor of Theoretical Physics\\
King's College London\\
Email: j.ellis@cern.ch
\end{minipage}

\end{resume}
\end{document}

